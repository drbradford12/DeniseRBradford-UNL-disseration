% From https://github.com/UWIT-IAM/UWThesis
\documentclass[print]{nuthesis}
\usepackage{amssymb, amsthm, amsmath, amsfonts}
\usepackage{wasysym}
\usepackage{mathrsfs}
% \usepackage{hyperref}
\usepackage{graphicx}
\usepackage{lineno}
\usepackage[colorinlistoftodos]{todonotes}
\usepackage{listings}
%\usepackage{breqn}
\usepackage{cancel, enumerate}
\usepackage{rotating, environ}
\usepackage{caption}
\usepackage{subcaption}
\usepackage[inline]{enumitem}
\usepackage{dirtree}

\newtheorem{thm}{Theorem}
\newtheorem{defn}{Definition}
\newtheorem{prop}{Proposition}
\newtheorem{lemma}{Lemma}
\newtheorem{cor}{Corollary}

% Syntax highlighting #22

%% https://github.com/rstudio/rmarkdown/issues/1649
\newlength{\cslhangindent}
\setlength{\cslhangindent}{1.5em}
\newenvironment{CSLReferences}[2]%
{\setlength{\parindent}{0pt}%
\everypar{\setlength{\hangindent}{\cslhangindent}}\ignorespaces}%
{\par}

% fix for pandoc 1.14
\providecommand{\tightlist}{%
  \setlength{\itemsep}{0pt}\setlength{\parskip}{0pt}}

%% something about tables, from https://github.com/ismayc/thesisdown/issues/122
\usepackage{calc}

%% for copyright symbol
\usepackage{textcomp}

%% to allow to rotate pages to landscape
\usepackage{lscape}
%% to adjust table column width
\usepackage{tabularx}

% suppress bottom page numbers on first page of each chapter
% because they overlap with text
\usepackage{etoolbox}
\patchcmd{\chapter}{plain}{empty}{}{}

%% for more attractive tables
\usepackage{booktabs}
\usepackage{longtable}


\usepackage{graphicx}


% Double spacing, if you want it.
\def\dsp{\def\baselinestretch{2.0}\large\normalsize}
% \dsp

% If the Grad. Division insists that the first paragraph of a section
% be indented (like the others), then include this line:
\usepackage{indentfirst}

%%%%%%%%%%%%%%%%%%
% If you want to use "sections" to partition your thesis
% un-comment the following:
%
% \counterwithout{section}{chapter}
% \setsecnumdepth{subsubsection}
% \def\sectionmark#1{\markboth{#1}{#1}}
% \def\subsectionmark#1{\markboth{#1}{#1}}
% \renewcommand{\thesection}{\arabic{section}}
% \renewcommand{\thesubsection}{\thesection.\arabic{subsection}}
% \makeatletter
% \let\l@subsection\l@section
% \let\l@section\l@chapter
% \makeatother

\renewcommand{\thetable}{\arabic{table}}
\renewcommand{\thefigure}{\arabic{figure}}

%%%%%%%%%%%%%%%%%%


%% Stuff from https://github.com/suchow/Dissertate

% The following line would print the thesis in a postscript font

% \usepackage{natbib}
% \def\bibpreamble{\protect\addcontentsline{toc}{chapter}{Bibliography}}

\setcounter{tocdepth}{1} % Print the chapter and sections to the toc
% controls depth of table of contents (toc): 0 = chapter, 1 = section, 2 = subsection

\usepackage{natbib}


% commands and environments needed by pandoc snippets
% extracted from the output of `pandoc -s`
%% Make R markdown code chunks work
\usepackage{array}
\usepackage{amssymb,amsmath}
\usepackage{ifxetex,ifluatex}
\ifxetex
  \usepackage{fontspec,xltxtra,xunicode}
  \defaultfontfeatures{Mapping=tex-text,Scale=MatchLowercase}
\else
  \ifluatex
    \usepackage{fontspec}
    \defaultfontfeatures{Mapping=tex-text,Scale=MatchLowercase}
  \else
    \usepackage[utf8]{inputenc}
  \fi
\fi
\usepackage{color}
\usepackage{fancyvrb}


\ifxetex
  \usepackage[setpagesize=false, % page size defined by xetex
              unicode=false, % unicode breaks when used with xetex
              xetex,
              colorlinks=true,
              linkcolor=blue]{hyperref}
\else
  \usepackage[unicode=true,
              colorlinks=true,
              linkcolor=blue]{hyperref}
\fi
\hypersetup{breaklinks=true, pdfborder={0 0 0}}
\setlength{\parindent}{20pt}
\setlength{\parskip}{6pt plus 2pt minus 1pt}
\setlength{\emergencystretch}{3em}  % prevent overfull lines
\setcounter{secnumdepth}{2} %% controls section numbering, e.g. 1 or 1.2, or 1.2.3



%  ----  Text Colors  ------------------------------------------
%
% Assign colors to writers for review
%
\newcommand{\db}[1]{{\textcolor{blue}{#1}}}
\newcommand{\svp}[1]{{\textcolor{RedOrange}{#1}}}
\newcommand{\cochaircol}[1]{{\textcolor{Green}{#1}}}
%


%  --- Code chunk font size -----------------------------------------------
% https://stackoverflow.com/questions/38323331/code-chunk-font-size-in-beamer-with-knitr-and-latexhttps://stackoverflow.com/questions/38323331/code-chunk-font-size-in-beamer-with-knitr-and-latex

%% change fontsize of R code
\let\oldShaded\Shaded
\let\endoldShaded\endShaded
\renewenvironment{Shaded}{\footnotesize\oldShaded}{\endoldShaded}

%% change fontsize of output
\let\oldverbatim\verbatim
\let\endoldverbatim\endverbatim
\renewenvironment{verbatim}{\footnotesize\oldverbatim}{\endoldverbatim}


\begin{document}
% \linenumbers{}
%% Start formatting the first few special pages
%% frontmatter is needed to set the page numbering correctly
\frontmatter
%% from thesisdown
% To pass between YAML and LaTeX the dollar signs are added by CII
\title{DATA SCIENCE, DASHBOARDS, AND THE WAY IT WORKS WITH STATISTICS}
\author{Denise Renee Bradford}
\adviser{Susan R. VanderPlas, Ph.D}
% \adviserAbstract{}
\major{Statistics}
\degreemonth{Month}
\degreeyear{Year}
% \copyrightpage
%%
%% For most people the defaults will be correct, so they are commented
%% out. To manually set these, just uncomment and make the needed
%% changes.
%% \college{Your college}
%% \city{Your City}
%%
%% For most people the following can be changed with a class
%% option. To manually set these, just uncomment the following and
%% make the needed changes.
%% \doctype{Thesis or Dissertation}
%% \degree{Your degree}
%% \degreeabbreviation{Your degree abbr.}
%%
%% Now that we know everything we need, we can generate the title page
%% itself.
%%
\maketitle


\begin{abstract}
    Here is my abstract. \emph{(350 word limit)}
\end{abstract}

%% Optional
\begin{copyrightpage}
\end{copyrightpage}

%% Optional
\begin{dedication}
Dedicated to\ldots{}
\end{dedication}

%%%%%%%%%%%%%%%%%%%
% Acknowledgments
%%%%%%%%%%%%%%%%%%%
\begin{acknowledgments}
Thank you to all my people!
\end{acknowledgments}
%%%%%%%%%%%%%%%%%%%

%%%%%%%%%%%%%%%%%%%
% Grant Information
%%%%%%%%%%%%%%%%%%%
% \begin{grantinfo}
%     % Add any grant info here
% \end{grantinfo}

%%%%%%%%%%%%%%%%%%%
% ToC
%%%%%%%%%%%%%%%%%%%
\tableofcontents

%%%%%%%%%%%%%%%%%%%
% List of Figures
%%%%%%%%%%%%%%%%%%%
\listoffigures
\listoftables

%%%%%%%%%%%%%%%%%%%
% Start of the document
%%%%%%%%%%%%%%%%%%%
\mainmatter


\hypertarget{unl-thesis-fields}{%
\chapter{UNL thesis fields}\label{unl-thesis-fields}}

Placeholder

\hypertarget{introduction}{%
\chapter{Introduction}\label{introduction}}

Placeholder

\hypertarget{audience-considerations}{%
\section{Audience Considerations}\label{audience-considerations}}

\hypertarget{perception}{%
\subsection{Perception}\label{perception}}

\hypertarget{attention-and-memory}{%
\subsection{Attention and Memory}\label{attention-and-memory}}

\hypertarget{constructing-meaning}{%
\subsection{Constructing Meaning}\label{constructing-meaning}}

\hypertarget{expertise}{%
\subsection{Expertise}\label{expertise}}

\hypertarget{engagement-with-the-data}{%
\subsection{Engagement with the data}\label{engagement-with-the-data}}

\hypertarget{data-considerations}{%
\section{Data Considerations}\label{data-considerations}}

\hypertarget{interactive-graphics}{%
\subsubsection{Interactive Graphics}\label{interactive-graphics}}

\hypertarget{audience-data-interactions}{%
\section{Audience-Data Interactions}\label{audience-data-interactions}}

\hypertarget{testing-static-graphics}{%
\subsection{Testing static graphics}\label{testing-static-graphics}}

\hypertarget{testing-interactive-graphics}{%
\subsection{Testing interactive graphics}\label{testing-interactive-graphics}}

\hypertarget{dashboard-design}{%
\section{Dashboard Design}\label{dashboard-design}}

\hypertarget{conclusion}{%
\section{Conclusion}\label{conclusion}}

\hypertarget{rmd-basics}{%
\chapter{Chapter Paper on Rural Shrink Smart Manuscript submitted to Journal of Data Science Special Issue}\label{rmd-basics}}

Placeholder

\hypertarget{abstract}{%
\section{Abstract}\label{abstract}}

\hypertarget{introduction-1}{%
\section{Introduction}\label{introduction-1}}

\hypertarget{data-description}{%
\section{Data Description}\label{data-description}}

\hypertarget{dashboard-design-considerations}{%
\section{Dashboard Design Considerations}\label{dashboard-design-considerations}}

\hypertarget{guiding-design-principles}{%
\section{Guiding Design Principles}\label{guiding-design-principles}}

\hypertarget{dashboard-design-process}{%
\section{Dashboard Design Process}\label{dashboard-design-process}}

\hypertarget{dashboard-components}{%
\subsection{Dashboard Components}\label{dashboard-components}}

\hypertarget{initial-draft}{%
\subsection{Initial Draft}\label{initial-draft}}

\hypertarget{redesign}{%
\subsection{Redesign}\label{redesign}}

\hypertarget{discussion}{%
\section{Discussion}\label{discussion}}

\hypertarget{future-work}{%
\section{Future Work}\label{future-work}}

\hypertarget{conclusions}{%
\section{Conclusions}\label{conclusions}}

\hypertarget{math-sci}{%
\chapter{Chapter 2 Stuff}\label{math-sci}}

\hypertarget{many-useful-visualizations-dont-have-easy-interactivity}{%
\section{Many useful visualizations don't have easy interactivity}\label{many-useful-visualizations-dont-have-easy-interactivity}}

\hypertarget{ref-labels}{%
\chapter{Tables, Graphics, References, and Labels}\label{ref-labels}}

\hypertarget{dashboard-eda}{%
\section{Dashboard EDA}\label{dashboard-eda}}

\hypertarget{conclusion-1}{%
\chapter*{Conclusion}\label{conclusion-1}}
\addcontentsline{toc}{chapter}{Conclusion}

If we don't want Conclusion to have a chapter number next to it, we can add the \texttt{\{-\}} attribute.

\textbf{More info}

And here's some other random info: the first paragraph after a chapter title or section head \emph{shouldn't be} indented, because indents are to tell the reader that you're starting a new paragraph. Since that's obvious after a chapter or section title, proper typesetting doesn't add an indent there.

\appendix

\hypertarget{the-first-appendix}{%
\chapter{The First Appendix}\label{the-first-appendix}}

This first appendix includes all of the R chunks of code that were hidden throughout the document (using the \texttt{include\ =\ FALSE} chunk tag) to help with readibility and/or setup.

\textbf{In the main Rmd file}

\textbf{In Chapter \ref{ref-labels}:}

\hypertarget{the-second-appendix-for-fun}{%
\chapter{The Second Appendix, for Fun}\label{the-second-appendix-for-fun}}

\hypertarget{colophon}{%
\chapter*{Colophon}\label{colophon}}
\addcontentsline{toc}{chapter}{Colophon}

This document is set in \href{https://github.com/georgd/EB-Garamond}{EB Garamond}, \href{https://github.com/adobe-fonts/source-code-pro/}{Source Code Pro} and \href{http://www.latofonts.com/lato-free-fonts/}{Lato}. The body text is set at 11pt with \(\familydefault\).

It was written in R Markdown and \(\LaTeX\), and rendered into PDF using \href{https://github.com/benmarwick/huskydown}{huskydown} and \href{https://github.com/rstudio/bookdown}{bookdown}.

This document was typeset using the XeTeX typesetting system, and the \href{http://staff.washington.edu/fox/tex/}{University of Washington Thesis class} class created by Jim Fox. Under the hood, the \href{https://github.com/UWIT-IAM/UWThesis}{University of Washington Thesis LaTeX template} is used to ensure that documents conform precisely to submission standards. Other elements of the document formatting source code have been taken from the \href{https://github.com/stevenpollack/ucbthesis}{Latex, Knitr, and RMarkdown templates for UC Berkeley's graduate thesis}, and \href{https://github.com/suchow/Dissertate}{Dissertate: a LaTeX dissertation template to support the production and typesetting of a PhD dissertation at Harvard, Princeton, and NYU}

The source files for this thesis, along with all the data files, have been organised into an R package, xxx, which is available at \url{https://github.com/xxx/xxx}. A hard copy of the thesis can be found in the University of Washington library.

This version of the thesis was generated on 2023-04-12 20:39:33. The repository is currently at this commit:

The computational environment that was used to generate this version is as follows:

\begin{verbatim}
## - Session info ---------------------------------------------------------------
##  setting  value
##  version  R version 4.2.2 (2022-10-31)
##  os       macOS Big Sur ... 10.16
##  system   x86_64, darwin17.0
##  ui       X11
##  language (EN)
##  collate  en_US.UTF-8
##  ctype    en_US.UTF-8
##  tz       America/New_York
##  date     2023-04-12
##  pandoc   2.19.2 @ /Applications/RStudio.app/Contents/Resources/app/quarto/bin/tools/ (via rmarkdown)
## 
## - Packages -------------------------------------------------------------------
##  package     * version date (UTC) lib source
##  bookdown      0.33    2023-03-06 [1] CRAN (R 4.2.0)
##  cachem        1.0.7   2023-02-24 [1] CRAN (R 4.2.0)
##  callr         3.7.3   2022-11-02 [1] CRAN (R 4.2.0)
##  cli           3.6.0   2023-01-09 [1] CRAN (R 4.2.0)
##  crayon        1.5.2   2022-09-29 [1] CRAN (R 4.2.0)
##  devtools      2.4.5   2022-10-11 [1] CRAN (R 4.2.0)
##  digest        0.6.31  2022-12-11 [1] CRAN (R 4.2.0)
##  ellipsis      0.3.2   2021-04-29 [1] CRAN (R 4.2.0)
##  evaluate      0.20    2023-01-17 [1] CRAN (R 4.2.0)
##  fastmap       1.1.1   2023-02-24 [1] CRAN (R 4.2.0)
##  fs            1.6.1   2023-02-06 [1] CRAN (R 4.2.0)
##  glue          1.6.2   2022-02-24 [1] CRAN (R 4.2.0)
##  htmltools     0.5.4   2022-12-07 [1] CRAN (R 4.2.0)
##  htmlwidgets   1.6.2   2023-03-17 [1] CRAN (R 4.2.0)
##  httpuv        1.6.9   2023-02-14 [1] CRAN (R 4.2.0)
##  knitr         1.42    2023-01-25 [1] CRAN (R 4.2.0)
##  later         1.3.0   2021-08-18 [1] CRAN (R 4.2.0)
##  lifecycle     1.0.3   2022-10-07 [1] CRAN (R 4.2.0)
##  magrittr      2.0.3   2022-03-30 [1] CRAN (R 4.2.0)
##  memoise       2.0.1   2021-11-26 [1] CRAN (R 4.2.0)
##  mime          0.12    2021-09-28 [1] CRAN (R 4.2.0)
##  miniUI        0.1.1.1 2018-05-18 [1] CRAN (R 4.2.0)
##  pkgbuild      1.4.0   2022-11-27 [1] CRAN (R 4.2.0)
##  pkgload       1.3.2   2022-11-16 [1] CRAN (R 4.2.0)
##  prettyunits   1.1.1   2020-01-24 [1] CRAN (R 4.2.0)
##  processx      3.8.0   2022-10-26 [1] CRAN (R 4.2.0)
##  profvis       0.3.7   2020-11-02 [1] CRAN (R 4.2.0)
##  promises      1.2.0.1 2021-02-11 [1] CRAN (R 4.2.0)
##  ps            1.7.4   2023-04-02 [1] CRAN (R 4.2.0)
##  purrr         1.0.1   2023-01-10 [1] CRAN (R 4.2.0)
##  R6            2.5.1   2021-08-19 [1] CRAN (R 4.2.0)
##  Rcpp          1.0.10  2023-01-22 [1] CRAN (R 4.2.0)
##  remotes       2.4.2   2021-11-30 [1] CRAN (R 4.2.0)
##  rlang         1.1.0   2023-03-14 [1] CRAN (R 4.2.0)
##  rmarkdown     2.20    2023-01-19 [1] CRAN (R 4.2.2)
##  rstudioapi    0.14    2022-08-22 [1] CRAN (R 4.2.0)
##  sessioninfo   1.2.2   2021-12-06 [1] CRAN (R 4.2.0)
##  shiny         1.7.4   2022-12-15 [1] CRAN (R 4.2.0)
##  stringi       1.7.12  2023-01-11 [1] CRAN (R 4.2.0)
##  stringr       1.5.0   2022-12-02 [1] CRAN (R 4.2.0)
##  urlchecker    1.0.1   2021-11-30 [1] CRAN (R 4.2.0)
##  usethis       2.1.6   2022-05-25 [1] CRAN (R 4.2.0)
##  vctrs         0.6.1   2023-03-22 [1] CRAN (R 4.2.0)
##  xfun          0.37    2023-01-31 [1] CRAN (R 4.2.0)
##  xtable        1.8-4   2019-04-21 [1] CRAN (R 4.2.0)
##  yaml          2.3.7   2023-01-23 [1] CRAN (R 4.2.0)
## 
##  [1] /Library/Frameworks/R.framework/Versions/4.2/Resources/library
## 
## ------------------------------------------------------------------------------
\end{verbatim}

\hypertarget{references}{%
\chapter*{References}\label{references}}
\addcontentsline{toc}{chapter}{References}

Placeholder

\hypertarget{testing-graphics}{%
\chapter{Testing Graphics}\label{testing-graphics}}

Placeholder

\hypertarget{variable-types}{%
\subsection{Variable Types}\label{variable-types}}

\hypertarget{multivariate-data-displays}{%
\subsection{Multivariate Data Displays}\label{multivariate-data-displays}}

\hypertarget{ggpcp-package---generalized-parallel-coordinate-plots}{%
\section{\texorpdfstring{\texttt{ggpcp} package - Generalized Parallel Coordinate Plots}{ggpcp package - Generalized Parallel Coordinate Plots}}\label{ggpcp-package---generalized-parallel-coordinate-plots}}

\hypertarget{parallel-coordinate-plot-visualization-review}{%
\section{Parallel Coordinate Plot Visualization Review}\label{parallel-coordinate-plot-visualization-review}}

\hypertarget{parallel-coordinates-in-eda}{%
\subsection{Parallel Coordinates in EDA}\label{parallel-coordinates-in-eda}}

\hypertarget{r-packages}{%
\subsection{R Packages}\label{r-packages}}

\hypertarget{ggpcp-package-importance}{%
\subsection{\texorpdfstring{\texttt{ggpcp} package importance}{ggpcp package importance}}\label{ggpcp-package-importance}}

\hypertarget{context-and-background}{%
\chapter{CONTEXT AND BACKGROUND}\label{context-and-background}}

Topic: The relationship between good statistical graphics and appropriate dashboard design.

\textbf{What is the current situation?}

\begin{itemize}
\tightlist
\item
  Data Scientists and Data Analysts are responsible for making decisions using large data sources for organizations. Large Data sources in decision-making have become more of the norm, which means that a consistent choice in dashboard design helps in exploratory data analysis (EDA). Limited resources and knowledge about the impact of large data sources can contribute to confusion and misconceptions about using large data in decision-making.
\end{itemize}

\textbf{What is wrong with the situation and/or what needs to change?}

\begin{itemize}
\tightlist
\item
  Many Dashboards are built to assist in data exploration, but many dashboards lack guidelines and appropriate perception with static and interactive statistical graphics.
\end{itemize}

\textbf{What are the most recent events that led us to the current situation?}

\textbf{What is the setting of the problem?}

\begin{itemize}
\tightlist
\item
  Dashboards with statistical graphics
\end{itemize}

\hypertarget{knowledge-gap}{%
\chapter{KNOWLEDGE GAP}\label{knowledge-gap}}

\textbf{What have researchers in similar settings explored or discovered?}

\textbf{What have researchers in other settings explored or discovered?}

\textbf{What remains unknown about the problem?}

\begin{itemize}
\tightlist
\item
  While it is known that interactive graphics is an essential aspect of statistical graphics and are used in dashboard designs frequently, the inconsistency gap between interactive graphics and dashboard design, it is unknown whether there is a relationship between interactive graphics and effective dashboard design can be implemented in Exploratory Data Analysis (EDA).
\end{itemize}

\hypertarget{significance}{%
\chapter{SIGNIFICANCE}\label{significance}}

\textbf{Why is it important to fill the knowledge gap?}

\begin{itemize}
\tightlist
\item
  All dashboards will benefit from a comprehensive guidelines for effective dashboard design.
\end{itemize}

\textbf{Scope: How big is the problem, and how many people are affected in the field or specific context you are studying?}

\begin{itemize}
\tightlist
\item
\end{itemize}

\textbf{Consequences: What are the consequences of not addressing the problem?}

\begin{itemize}
\tightlist
\item
  Dashboards will continue to be designed without the appropriate perception of good statistical interactive graphics when analysts develop dashboards for exploring large data.
\end{itemize}

\textbf{Contribution: How will your study contribute to the problem's solution? How does filling the gap help?}

\hypertarget{draft-problem-statement}{%
\chapter{Draft Problem Statement}\label{draft-problem-statement}}


%% backmatter is needed at the end of the main body of your thesis to
%% set up page numbering correctly for the remainder of the thesis
\backmatter

%% Start the correct formatting for the appendices
% \appendix
%% Input each appendix here
% \input{./appendix_a}

%% Bibliography goes here (You better have one)
%% BibTeX is your friend

% \bibliographystyle{alpha}  % or use  abbrv to abbreviate first names and use numerical indices
\bibliographystyle{abbrv}  % or use  abbrv to abbreviate first names and use numerical indices
%% Add your BibTex file here (don't include the .bib)
\bibliography{./references}



%% Index go here (if you have one)
\end{document}
